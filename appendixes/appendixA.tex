\appendix
\chapter{Tools}
This appendix describes the design and construction of lock picking tools.

\section{Pick Shapes}
Picks come in several shapes and sizes. Figure A.l shows the most common shapes. The 
handle and tang of a pick are the same for all picks. The handle must be comfortable and 
the tang must be thin enough to avoid bumping pins unnecessarily. If the tang is too thin, 
then it will act like a spring and you will loose the feel of the tip interacting with the pins. 
The shape of the tip determines how easily the pick passes over the pins and what kind of 
feedback you get from each pin. 

The design of a tip is a compromise between ease of insertion, ease of withdrawal and feel 
of the interaction. The half diamond tip with shallow angles is easy to insert and remove, 
so you can apply pressure when the pick is moving in either direction. It can quickly pick a 
lock that has little variation in the lengths of the key pins. If the lock requires a key that 
has a deep cut between two shallow cuts, the pick may not be able to push the middle pin 
down far enough. The half diamond pick with steep angles could deal with such a lock, and 
in general steep angles give you better feedback about the pins. Unfortunately, the steep 
angles make it harder to move the pick in the lock. A tip that has a shallow front angle and 
a steep back angle works well for Yale locks. 

The half round tip works well in disk tumbler locks. See section 9.13. The full diamond 
and full round tips are useful for locks that have pins at the top and bottom of the keyway. 

The rake tip is designed for picking pins one by one. It can also be used to rake over 
the pins, but the pressure can only be applied as the pick is withdrawn. The rake tip allows 
you to carefully feel each pin and apply varying amounts of pressure. Some rake tips are flat 
or dented on the top to makes it easier to align the pick on the pin. The primary benefit 
of picking pins one at a time is that you avoid scratching the pins. Scrubbing scratches the 
tips of the pins and the keyway, and it spreads metal dust throughout the lock. If you want 
to avoid leaving traces, you must avoid scrubbing. 

The snake tip can be used for scrubbing or picking. When scrubbing, the multiple bumps generate 
more action than a regular pick. The snake tip is particularly good at opening five pin household locks. 
When a snake tip is used for picking, it can set two or three pins at 
once. Basically, the snake pick acts like a segment of a key which can be adjusted by lifting 
and lowering the tip, by tilting it back and forth, and by using either to top or bottom of 
the tip. You should use moderate to heavy torque with a snake pick to allow several pins to 
bind at the same time. This style of picking is faster than using a rake and it leaves as little 
evidence.

\section{Street cleaner bristles}
The spring steel bristles used on street cleaners make excellent tools for lock picking. The 
bristles have the right thickness and width, and they are easy to grind into the desired shape. 
The resulting tools are springy and strong. Section A. 3 describes how to make tools that 
are less springy. 

The first step in making tools is to sand off any rust on the bristles. Course grit sand 
paper works fine as does a steel wool cleaning pad (not copper wool). If the edges or tip of 
the bristle are worn down, use a file to make them square. 

A torque wrench has a head and a handle as shown in figure A. 2. The head is usually 
1/2 to 3/4 of an inch long and the handle varies from 2 to 4 inches long. The head and the 
handle are separated by a bend that is about 80 degrees. The head must be long enough 
to reach over any protrusions (such as a grip-proof collar) and firmly engage the plug. A 
long handle allows delicate control over the torque, but if it is too long, it will bump against 
the doorframe. The handle, head and bend angle can be made quite small if you want to 
make tools that are easy to conceal (e.g., in a pen, flashlight, or belt buckle). Some torque 
wrenches have a 90 degree twist in the handle. The twist makes it easy to control the torque 
by controlling how far the handle has been deflected from its rest position. The handle acts 
as a spring which sets the torque. The disadvantage of this method of setting the torque is 
that you get less feedback about the rotation of the plug. To pick difficult locks you will 
need to learn how to apply a steady torque via a stiff handled torque wrench. 

The width of the head of a torque wrench determines how well it will fit the keyway. 
Locks with narrow keyways (e.g., desk locks) need torque wrenches with narrow heads. 
Before bending the bristle, file the head to the desired width. A general purpose wrench can 
be made by narrowing the tip (about 1/4 inch) of the head. The tip fits small keyways while 
the rest of the head is wide enough to grab a normal keyway. 

The hard part of making a torque wrench is bending the bristle without cracking it. To 
make the 90 degree handle twist, clamp the head of the bristle (about one inch) in a vise 
and use pliers to grasp the bristle about 3/8 of an inch above the vise. You can use another 
pair of pliers instead of a vise. Apply a 45 degree twist. Try to keep the axis of the twist 
lined up with the axis of the bristle. Now move the pliers back another 3/8 inch and apply 
the remaining 45 degrees. You will need to twist the bristle more than 90 degrees in order 
to set a permanent 90 degree twist. 

To make the 80 degree head bend, lift the bristle out of the vise by about 1/4 inch (so 
3/4 inch is still in the vise). Place the shank of a screw driver against the bristle and bend 
the spring steel around it about 90 degrees. This should set a permanent 80 degree bend in 
the metal. Try to keep the axis of the bend perpendicular to the handle. The screwdriver 
shank ensures that the radius of curvature will not be too small. Any rounded object will 
work (e.g., drill bit, needle nose pliers, or a pen cap). If you have trouble with this method, 
try grasping the bristle with two pliers separated by about 1/2 inch and bend. This method 
produces a gentle curve that won't break the bristle. 

A grinding wheel will greatly speed the job of making a pick. It takes a bit of practice 
to learn how make smooth cuts with a grinding wheel, but it takes less time to practice and 
make two or three picks than it does to hand file a single pick. The first step is to cut the 
front angle of the pick. Use the front of the wheel to do this. Hold the bristle at 45 degrees 
to the wheel and move the bristle side to side as you grind away the metal. Grind slowly 
to avoid overheating the metal, which makes it brittle. If the metal changes color (to dark 
blue), you have overheated it, and you should grind away the colored portion. Next, cut 
the back angle of the tip using the corner of the wheel. Usually one corner is sharper than 
the other, and you should use that one. Hold the pick at the desired angle and slowly push 
it into the corner of the wheel. The side of the stone should cut the back angle. Be sure 
that the tip of the pick is supported. If the grinding wheel stage is not close enough to the 
wheel to support the tip, use needle nose pliers to hold the tip. The cut should should pass 
though about 2/3 of the width of the bristle. If the tip came out well, continue. Otherwise 
break it off and try again. You can break the bristle by clamping it into a vise and bending 
it sharply. 

The corner of the wheel is also used to grind the tang of the pick. Put a scratch mark 
to indicate how far back the tang should go. The tang should be long enough to allow the 
tip to pass over the back pin of a seven pin lock. Cut the tang by making several smooth 
passes over the corner. Each pass starts at the tip and moves to the scratch mark. Try to 
remove less than a l/16th of an inch of metal with each pass. I use two fingers to hold the 
bristle on the stage at the proper angle while my other hand pushes the handle of the pick 
to move the tang along the corner. Use whatever technique works best for you. 

Use a hand file to finish the pick. It should feel smooth if you run a finger nail over it. 
Any roughness will add noise to the feedback you want to get from the lock. 

The outer sheath of phone cable can be used as a handle for the pick. Remove three or 
four of the wires from a length of cable and push it over the pick. If the sheath won't stay 
in place, you can put some epoxy on the handle before pushing the sheath over it. 

\section{Bicycle spokes}
An alternative to making tools out of street cleaner bristles is to make them out of nails and 
bicycle spokes. These materials are easily accessible and when they are heat treated, they 
will be stronger than tools made from bristles. 

A strong torque wrench can be constructed from an 8-penny nail (about .1 inch diameter). 
First heat up the point with a propane torch until it glows red, slowly remove it from the
flame, and let it air cool; this softens it. The burner of a gas stove can be used instead of 
a torch. Grind it down into the shape of a skinny screwdriver blade and bend it to about 
80 degrees. The bend should be less than a right angle because some lock faces are recessed 
behind a plate (called an escutcheon) and you want the head of the wrench to be able to 
reach about half an inch into the plug. Temper (harden) the torque wrench by heating to 
bright orange and dunking it into ice water. You will wind up with a virtually indestructible 
bent screwdriver that will last for years under brutal use. 

Bicycle spokes make excellent picks. Bend one to the shape you want and file the sides of 
the business end flat such that it's strong in the vertical and flexy in the horizontal direction. 
Try a right-angle hunk about an inch long for a handle. For smaller picks, which you need 
for those really tiny keyways, find any large-diameter spring and unbend it. If you're careful 
you don't have to play any metallurgical games.

\section{Brick Strap}
For perfectly serviceable key blanks that you can't otherwise find at the store, use the metal 
strap they wrap around bricks for shipping. It's wonderfully handy stuff for just about 
anything you want to manufacture. To get around side wards in the keyway, you can bend 
the strap lengthwise by clamping it in a vice and tapping on the protruding part to bend 
the piece to the required angle. 

Brick strap is very hard. It can ruin a grinding wheel or key cutting machine. A hand 
file is the recommended tool for milling brick strap. 
