\appendix
\chapter{Legal Issues}

Contrary to widespread myth, it is not a felony to possess lockpicks. Each state has its own 
laws with respect to such burglarious instruments. Here is the Massachusetts version quoted 
in entirety from the massachusetts general code: 


Chapter 266 (crimes against property) 
Section 49. Burglarious instruments; making; possession; use. 

Whoever makes or mends, or begins to make or mend, or knowingly has in 
his possession, an engine, machine, tool or implement adapted and designed for 
cutting through, forcing or breaking open a building, room, vault, safe or other 
depository, IN ORDER TO STEAL THEREFROM money or other property, or 
to commit any other crime, knowing the same to be adapted and designed for 
the purpose aforesaid, WITH INTENT TO USE OR EMPLOY OR ALLOW 
the same to be used or employed for such purpose, or whoever knowingly has in 
his possession a master key designed to fit more than one motor vehicle, WITH 
INTENT, TO USE OR EMPLOY THE SAME to steal a motor vehicle or other 
property therefrom, shall be punished by imprisonment in the state prison for 
not more than ten years or by a fine of not more than one thousand dollars and 
imprisonment in jail for not more than two and one half years. 

Emphasis added. 

In other words, mere possession means nothing. If they stop you for speeding or something, 
and find a pick set, they can't do much. On the other hand, if they catch you picking 
the lock on a Monec machine they get to draw and quarter you. 

States with similar wording include ME, NH, NY. One place that DOES NOT have 
similar wording, and does make possession illegal, is Washington, DC. These are the only 
other places I have checked. I would imagine that most states are similar to Massachusetts, 
but I would not bet anything substantial (say, more than a slice of pizza) on it. 

It may be a good idea to carry around a xeroxed copy of the appropriate page from your 
state's criminal code.
