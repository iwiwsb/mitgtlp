\chapter{The Pin Column Model}
The flatland model of locks can explain effects that involve more than one pin, but a different
model is needed to explain the detailed behavior of a single pin. See Figure 5.1.
The pin-column model highlights the relationship between the torque applied and the amount of force
needed to lift each pin. It is essential that you understand this relationship.

In order to understand the "feel" of lock picking you need to know how the movement
of a pin is effect by the torque applied by your torque wrench (tensioner) and the pressure
applied by your pick. A good way to represent this understanding is a graph that shows the
minimum pressure needed to move a pin as a function of how far the pin has been displaced
from its initial position. The remainder of this chapter will derive that force graph from the
pin-column model.

Figure 5.2 shows a single pin position after torque has been applied to the plug. The
forces acting of the driver pin are the friction from the sides, the spring contact force from
above, and the contact force from the key pin below. The amount of pressure you apply to
the pick determines the contact force from below.
The spring force increases as the pins are pushed into the hull, but the increase is slight,
so we will assume that the spring force is constant over the range of displacements we
are interested in. The pins will not move unless you apply enough pressure to overcome
the spring force. The binding friction is proportional to how hard the driver pin is being
scissored between the plug and the hull, which in this case is proportional to the torque. The
more torque you apply to the plug, the harder it will be to move the pins. To make a pin
move, you need to apply a pressure that is greater than the sum of the spring and friction
forces.

When the bottom of the driver pin reaches the sheer line, the situation suddenly changes.
See Figure 5.3. The friction binding force drops to zero and the plug rotates slightly (until
some other pin binds). Now the only resistance to motion is the spring force. After the
top of the key pin crosses the gap between the plug and the hull, a new contact force arises
from the key pin striking the hull. This force can be quite large, and it causes a peak in the
amount of pressure needed to move a pin.

If the pins are pushed further into the hull, the key pin acquires a binding fiction like the
driver pin had in the initial situation. See Figure 5.4. Thus, the amount of pressure needed
to move the pins before and after the sheer line is about the same. Increasing the torque
increases the required pressure. At the sheer line, the pressure increases dramatically due to
the key pin hitting the hull. This analysis is summarized graphically in Figure 5.5.
