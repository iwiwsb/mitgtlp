\chapter{It's easy}
The big secret of lock picking is that it's easy. Anyone can learn how to pick locks.

The theory of lock picking is the theory of exploiting mechanical defects. There are a
few basic concepts and definitions but the bulk of the material consists of tricks for opening
locks with particular defects or characteristics. The organization of this manual reflects
this structure. The first few chapters present the vocabulary and basic information about locks
and lock picking. There is no way to learn lock picking without practicing, so one chapter
presents a set of carefully chosen exercises that will help you learn the skills of lock picking.
The document ends with a catalog of the mechanical traits and defects found in locks and
the techniques used to recognize and exploit them. The first appendix describes how to make
lock picking tools. The other appendix presents some of the legal issues of lock picking.

The exercises are important. The only way to learn how to recognize and exploit the
defects in a lock is to practice. This means practicing many times on the same lock as well
as practicing on many different locks. Anyone can learn how to open desk and filing cabinet
locks, but the ability to open most locks in under thirty seconds is a skill that requires
practice.

Before getting into the details of locks and picking, it is worth pointing out that lock
picking is just one way to bypass a lock, though it does cause less damage than brute force
techniques. In fact, it may be easier to bypass the bolt mechanism than to bypass the lock.
It may also be easier to bypass some other part of the door or even avoid the door entirely.
Remember: There is always another way, usually a better one.
