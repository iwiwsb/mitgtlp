\chapter{Exercises}
This chapter presents a series of exercises that will help you learn the basic skill of lock 
picking. Some exercises teach a single skill, while others stress the coordination of skills. 

When you do these exercises, focus on the skills, not on opening the lock. If you focus 
on opening the lock, you will get frustrated and your mind will stop learning. The goal of 
each exercise is to learn something about the particular lock you are holding and something 
about yourself. If a lock happens to open, focus on the memory of what you were doing and 
what you felt just before it opened. 

These exercises should be practiced in short sessions. After about thirty minutes you 
will find that your fingers become sore and your mind looses its ability to achieve relaxed 
concentration. 

\section{Exercise 1: Bouncing the pick}
This exercise helps you learn the skill of applying a fixed pressure with the pick independent 
of how the pick moves up and down in the lock. Basically you want to learn how to let the 
pick bounce up and down according to the resistance offered by each pin. 

How you hold the pick makes a difference on how easy it is to apply a fixed pressure. 
You want to hold it in such a way that the pressure comes from your fingers or your wrist. 
Your elbow and shoulder do not have the dexterity required to pick locks. While you are 
scrubbing a lock notice which of your joints are fixed, and which are allowed to move. The 
moving joints are providing the pressure. 

One way to hold a pick is to use two fingers to provide a pivot point while another finger 
levers the pick to provide the pressure. Which fingers you use is a matter of personal choice. 
Another way to hold the pick is like holding a pencil. With this method, your wrist provides 
the pressure. If your wrist is providing the pressure, your shoulder and elbow should provide 
the force to move the pick in and out of the lock. Do not use your wrist to both move the 
pick and apply pressure. 

A good way to get used to the feel of the pick bouncing up and down in the keyway is to 
try scrubbing over the pins of an open lock. The pins cannot be pushed down, so the pick must 
adjust to the heights of the pins. Try to feel the pins rattle as the pick moves over 
them. If you move the pick quickly, you can hear the rattle. This same rattling feel will help 
you recognize when a pin is set correctly. If a pin appears to be set but it doesn't rattle, 
then it is false set. False set pins can be fixed by pushing them down farther, or by releasing 
torque and letting them pop back to their initial position. 

One last word of advice. Focus on the tip of the pick. Don't think about how you are 
moving the handle; think about how you are moving the tip of the pick.

\section{Exercise 2: Picking pressure}
This exercise will teach you the range of pressures you will need to apply with a pick. When 
you are starting, just apply pressure when you are drawing the pick out of the lock. Once 
you have mastered that, try applying pressure when the pick is moving inward. 

With the flat side of your pick, push down on the first pin of a lock. Don't apply any 
torque to the lock. The amount of pressure you are applying should be just enough to 
overcome the spring force. This force gives you an idea of minimum pressure you will apply 
with a pick. 

The spring force increases as you push the pin down. See if you can feel this increase. 

Now see how it feels to push down the other pins as you pull the pick out of the lock. 
Start out with both the pick and torque wrench in the lock, but don't apply any torque. As 
you draw the pick out of the lock, apply enough pressure to push each pin all the way down. 

The pins should spring back as the pick goes past them. Notice the sound that the pins 
make as they spring back. Notice the popping feel as a pick goes past each pin. Notice the 
springy feel as the pick pushes down on each new pin. 

To help you focus on these sensations, try counting the number of pins in the lock. Door 
locks at MIT have seven pins, padlocks usually have four. 

To get an idea of the maximum pressure, use the flat side of your pick to push down all 
the pins in the lock. Sometimes you will need to apply this much pressure to a single pin. 
If you encounter a new kind of lock, perform this exercise to determine the stiffness of its 
springs. 

\section{Exercise 3: Picking Torque}
This exercise will teach you the range of torque you will need to apply to a lock. It demon- 
strates the interaction between torque and pressure which was describe in chapter 5. 

The minimum torque you will use is just enough to overcome the fiction of rotating the 
plug in the hull. Use your torque wrench to rotate the plug until it stops. Notice how much 
torque is needed to move the plug before the pins bind. This force can be quite high for 
locks that have been left out in the rain. The minimum torque for padlocks includes the 
force of a spring that is attached between the plug and the shackle bolt. 

To get a feel for the maximum value of torque, use the flat side of the pick to push all 
the pins down, and try applying enough torque to make the pins stay down after the pick is 
removed. If your torque wrench has a twist in it, you may not be able to hold down more 
than a few pins. 

If you use too much torque and too much pressure you can get into a situation like the 
one you just created. The key pins are pushed too far into the hull and the torque is sufficient 
to hold them there. 

The range of picking torque can be found by gradually increasing the torque while scrub- 
bing the pins with the pick. Some of the pins will become harder to push down. Gradually 
increase the torque until some of the pins set. These pins will loose their springiness. Keep- 
ing the torque fixed, use the pick to scrub the pins a few times to see if other pins will 
set. 

The most common mistake of beginners is to use too much torque. Use this exercise to 
find the minimum torque required to pick the lock.

\section{Exercise 4: Identifying Set Pins}
While you are picking a lock, try to identify which pins are set. You can tell a pin is set 
because it will have a slight give. That is, the pin can be pushed down a short distance 
with a light pressure, but it becomes hard to move after that distance (see chapter 6 for an 
explanation). When you remove the light pressure, the pin springs back up slightly. Set pins 
also rattle if you flick them with the pick. Try listening for that sound. 

Run the pick over the pins and try to decide whether the set pins are in the front or back 
of the lock (or both). Try identifying exactly which pins are set. Remember that pin one is 
the frontmost pin (i.e., the pin that a key touches first). The most important skill of lock 
picking is the ability to recognize correctly set pins. This exercise will teach you that skill. 

Try repeating this exercise with the plug turning in the other direction. If the front pins 
set when the plug is turned one way, the back pins will set when the plug is turned the other 
way. See Figure 6.2 for an explanation. 

One way to verify how many pins are set is to release the torque, and count the clicks as 
the pins snap back to their initial position. Try this. Try to notice the difference in sound 
between the snap of a single pin and the snap of two pins at once. A pin that has been false 
set will also make a snapping sound. 

Try this exercise with different amounts of torque and pressure. You should notice that 
a larger torque requires a larger pressure to make pins set correctly. If the pressure is too 
high, the pins will be jammed into the hull and stay there.

\section{Exercise 5: Projection}
As you are doing the exercises try building a picture in your mind of what is going on. The 
picture does not have to be visual, it could be a rough understanding of which pins are set 
and how much resistance you are encountering from each pin. One way to foster this picture 
building is to try to remember your sensations and beliefs about a lock just before it opened. 
When a lock opens, don't think "that's over", think "what happened". 

This exercise requires a lock that you find easy to pick. It will help you refine the visual 
skills you need to master lock picking. Pick the lock, and try to remember how the process 
felt. Rehearse in your mind how everything feels when the lock is picked properly. Basically, 
you want to create a movie that records the process of picking the lock. Visualize the 
motion of your muscles as they apply the correct pressure and torque, and feel the resistance 
encountered by the pick. Now pick the lock again trying to match your actions to the movie. 

By repeating this exercise, you are learning how to formulate detailed commands for your 
muscles and how to interpret feedback from your senses. The mental rehearsal teaches you 
how to build a visual understanding of the lock and how to recognize the major steps of 
picking it.
