\chapter{Advanced Lock Picking}
Simple lock picking is a trade that anyone can learn. However, advanced lock picking is a craft
that requires mechanical sensitivity, physical dexterity, visual concentration and analytic
thinking. If you strive to excel at lock picking, you will grow in many ways.

\section{Mechanical Skills}
Learning how to pull the pick over the pins is surprisingly difficult. The problem is that the
mechanical skills you learned early in life involved maintaining a fixed position or fixed path
for your hands independent of the amount of force required. In lock picking, you must learn
how to apply a fixed force independent of the position of your hand. As you pull the pick
out of the lock you want to apply a fixed pressure on the pins. The pick should bounce up
and down in the keyway according to the resistance offered by each pin.

To pick a lock you need feedback about the effects of your manipulations. To get the
feedback, you must train yourself to be sensitive to the sound and feel of the pick passing
over the pins. This is a mechanical skill that can only be learned with practice. The exercises
will help you recognize the important information coming from your fingers.

\section{Zen and the Art of Lock Picking}
In order to excel at lock picking, you must train yourself to have a visually reconstructive
imagination. The idea is to use information from all your senses to build a picture of what
is happening inside the lock as you pick it. Basically, you want to pro ject your senses into
the lock to receive a full picture of how it is responding to your manipulations. Once you
have learned how to build this picture, it is easy to choose manipulations that will open the
lock.

All your senses provide information about the lock. Touch and sound provide the most
information, but the other senses can reveal critical information. For example, your nose
can tell you whether a lock has been lubricated recently. As a beginner, you will need to use your eyes for hand-eye coordination, but as you improve you will nd it unnecessary to look
at the lock. In fact, it is better to ignore your eyes and use your sight to build an image of
the lock based on the information you receive from your fingers and ears.

The goal of this mental skill is to acquire a relaxed concentration on the lock. Don't
force the concentration. Try to ignore the sensations and thoughts that are not related to
the lock. Don't try to focus on the lock

\section{Analytic Thinking}
Each lock has its own special characteristics which make picking harder or easier. If you
learn to recognize and exploit the "personality traits" of locks, picking will go much faster.
Basically, you want to analyze the feedback you get from a lock to diagnose its personality
traits and then use your experience to decide on an approach to open the lock. Chapter 9
discusses a large number of common traits and ways to exploit or overcome them.

People underestimate the analytic skills involved in lock picking. They think that the
picking tool opens the lock. To them the torque wrench is a passive tool that just puts the
lock under the desired stress. Let me propose another way to view the situation. The pick
is just running over the pins to get information about the lock. Based on an analysis that
information the torque is adjusted to make the pins set at the sheer line. It's the torque
wrench that opens the lock.

Varying the torque as the pick moves in and out of the keyway is a general trick that can
be used to get around several picking problems. For example, if the middle pins are set, but
the end pins are not, you can increase the torque as the pick moves over the middle pins.
This will reduce the chances of disturbing the correctly set pins. If some pin doesn't seem to
lift up far enough as the pick passes over it, then try reducing the torque on the next pass.

The skill of adjusting the torque while the pick is moving requires careful coordination
between your hands, but as you become better at visualizing the process of picking a lock,
you will become better at this important skill.
